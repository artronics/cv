\begin{tabularx}{\textwidth}{p{\textwidth/7} p{\textwidth*6/7}}
    Title & \textbf{Software-Defined Wireless Sensor Networks (SDWSN)} \\[1.3ex]
    Description& \small This project brings \textbf{Software-Defined Networking (SDN)} concepts in to \textbf{Wireless Sensor Networks (WSNs)}. A \textbf{Client-Server} model connects each separated WSN to a \textbf{Centralized Controller} located in Server. The Client-Server communication happens via \textbf{REST} APIs. A \textbf{Single Page Application (SPA)} written in javascript (Angularjs) is in charge of monitoring the whole network. \normalsize \\[1ex]
    Significance & 
 \footnotesize{\textbullet~Highly modular and loosely coupled design.}\\&
 \footnotesize{\textbullet~Clean and robust code; more than 200 unit tests with code coverage up to 80 percent}
 \\\multicolumn{2}{c}{} \\

    Title & \textsc{\textsc{Kaivan:}}\textbf{Equipment Maintenance Software}\\[1.3ex]
    Description &\small \textsc{Kaivan} is a small and easy to use \textbf{Computerized Maintenance Management System (CMMS)} written with \textbf{Excel} and \textbf{VBA}. It was developed for \emph{Fidar Saze Co} and it is still in use at \emph{Asaluyeh} in Oil Platforms. \normalsize\\[1ex]
    Significance &
    \footnotesize{\textbullet~Easy to use with familiar Excel environment}\\&
    \footnotesize{\textbuller~A comprehensive GUI written in VBA}
 \\\multicolumn{2}{c}{} \\

    Title & \textbf{Motion Controlled Robotic Arm}\\[1.3ex]
    Description & \small A \textbf{Servo} robotic arm which is controlled by special developed sensor attached to human arm. The embedded system utilizes an ARM Cortex-M3 board running uC/OS-II an sophisticated control algorithms.\normalsize\\[1ex]
    Significance &
    \footnotesize{\textbullet~Real-Time task scheduler with uC/OS}
 \\\multicolumn{2}{c}{} \\

    Title & \textbf{GSM Alarm Transmitter}\\[1.3ex]
    Description & \small An industrial GSM SMS Alarm Transmitter.\normalsize\\[1ex]
    Significance &
    \footnotesize{\textbullet~Programming and GUI design with \textbf{LabVIEW}}\\&
    \footnotesize{\textbullet~Serial interface with AT Commands.}
 \\\multicolumn{2}{c}{} \\

    Title & \textbf{Modular Microcontroller Evaluation Board}\\[1.3ex]
    Description & \small A self-interest project which helps a developer to learn various aspects of microcontroller and hardware design by plugging different modules. This project supports multiple 8-bit PIC and AVR Microcontrollers with four different daughter boards for working with different IO functionalities.\normalsize\\[1ex]
    Significance &
    \footnotesize{\textbullet~Fully modular and expandable design by utilizing open architecture}\\&
    \footnotesize{\textbullet~Supports different Microcontroller Families}

 \\\multicolumn{2}{c}{} \\
    
    Title & \textbf{CAN Controller Module}\\[1.3ex]
    Description & \small This project provides a flexible interface for expanding extra floors to elevator controll board. The main bord communicates with different modules via CAN protocol.\normalsize\\[1ex]
    Significance &
    \footnotesize{\textbullet~Cost-effective and maintainable solution for expandign current elevator infrastructure}\\&
    \footnotesize{\textbullet~\textbf{MCP2515} CAN Controller with \textbf{MCP2552} CAN Tranceiver}\\&
    \footnotesize{\textbullet~\textbf{PIC18F} microcontroller}
 \\\multicolumn{2}{c}{} \\

    Title & \textbf{Time and Attendance Management System}\\[1.3ex]
    Description & \small A self-interest simple Time and Attence System developed with AVR mega microcontroller.\normalsize\\[1ex]
    Significance &
    \footnotesize{\textbullet~A \textbf{menue system} implementation with state diagram technique}\\&
    \footnotesize{\textbullet~Support for multiple users with different access permissions}
 \\\multicolumn{2}{c}{} \\

    Title & \textbf{Senatorino: A cross platform SDWSN solution}\\[1.3ex]
    Description & \small A new born self-interest open-source project based on my M.Sc thesis, started on 23 MArch 2016. This project is based on my intrest for learning Linux Device Driver development. It will provide a cross platform 802.15.4 solution for both Linux and other architectures.\normalsize\\[1ex]
    Significance &
    \footnotesize{\textbullet~802.15.4 cross platform solution, currently based on \textbf{MRF24J} tranceiver under Linux platform}\\&
    \footnotesize{\textbuller~First platform targets: Raspberry pi (Linux) and Arduino (ATmega) platforms}
 \\\multicolumn{2}{c}{} \\

\end{tabularx}
